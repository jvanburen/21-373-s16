\documentclass[notes.tex]{subfiles}

\begin{document}

\lecture{6}{2016--01--27}

\begin{definition}
	The \kw{cardinality} (or \kw{order}) of a group $G$ is the number of elements in it, denoted by $|G|$.
\end{definition}

\begin{eg}\leavevmode
	\begin{itemize}
		\item  $|\ZZ| = \infty (= \aleph_0)$
		\item $|(\ZZ/5\ZZ)| = 5$
		\item $|S_4| = 4! = 24$
	\end{itemize}
\end{eg}


\begin{definition}
	Given a group $G$ and $x\in G$, the \kw{order} of $x$ is the smallest integer $n>0$ such that $x^n=e$. If no such $n$ exists, we say the order is $\infty$.

	We denote by $|x|$ the order of $x$.
\end{definition}

\begin{eg}
	In $(\ZZ, +)$:
	$|0| = 1$, $|5| = \infty$.

	In $S_5$:
	$|\cyc{1 3}| = 2$, $|\cyc{2 4 5}| = 3$, $|\cyc{1 3}\cyc{2 4 5}| = 6$.
\end{eg}

\begin{proposition}
	If $G$ is a \emph{finite}  group, every $x\in G$ has finite order.
	Moreover, $|x| \le |G|$.
\end{proposition}
\begin{proof}
	Say $|G| = k$. Consider the sequence $x^0, x^1, x^2, \ldots, x^k$.
	There are $k+1$ items in the sequence. So $\exists m < n$ such that $x^m=x^n$.
		$x^{n-m} = x^nx^{-m} = x^mx^{-m} = e$.
		As $0 < n-m \le k$, it follows that $|x|\le n-m \le |G|$.
\end{proof}

\chapter{Subgroups} % (fold)
\label{ssub:subgroups}
\begin{definition}
	Suppose $(G, *)$ is a group and $H\subseteq G$ some subset of $G$. 
	We say $H$ is a \kw{subgroup} of $G$, written $H\le G$ if $(H, *)$ happens to be a group, i.e., the following properties hold:
\end{definition}

$*$ is a associative binary operator on $H$ (i.e., it's closed) with inverses and an identity element.

\begin{eg}\leavevmode
	\begin{itemize}
		\item $\ZZ\le\QQ$ (under addition)
		\item Even integers $\le \ZZ$ (under addition)
		\item $n\ZZ \le\ZZ$, where $n\ZZ := \{nx : x\in \ZZ\}$

		Aside: every subgroup of $\ZZ$ is of the form $n\ZZ$
		\item $\{e, \cyc{1 2}\cyc{3 4}, \cyc{1 3}\cyc{2 4}, \cyc{1 4}\cyc{2 3}\} \le S_4$.
	\end{itemize}
\end{eg}


\begin{proposition}
	\label{prop:subgrouphw}
	(Homework): $H\le G$ iff
	\begin{enumerate}[(a)]
		\item $H\ne\vs$ (nonempty)
		\item $\forall x,y\in H (xy\in H)$ (closed under product)
		\item $\forall x\in H (\inv x\in H)$  (closed under inverses)
	\end{enumerate}
\end{proposition}

\begin{proposition}
	Suppose $G$ is a \emph{finite} group.
	Then $H \le G$ iff $H \ne\vs$ and $\forall x,y\in H: xy\in H$.
\end{proposition}
% subsubsection subgroups (end)

\begin{proof}
	We show that for $H\subseteq G$
	(a) and (b) $\implies$ (c) (letters from proposition \eqref{prop:subgrouphw})

	Fix $x\in H$.
	Since $G$ is finite, $|x|$ is finite (in $G$). Say $|x| = n > 0$
	$x^n = \underbrace{x\cdot x\cdot \ldots \cdot x}_{n\text{ times}} = e_G.$ Hence, $e_G\in H$.

	Examine $x^{n-1}$.

	$x^{n-1} = \begin{cases}
		x^0 = e & \text{if } n = 1\\
		\underbrace{x\cdot x\cdot\ldots\cdot x}_{n\text{ times}} &\text{if } n > 1
	\end{cases}$

	But $x^{n-1} = \inv x$, since $x^{n-1}x = x^n = xx^{n-1} = e$.
	Thus, (c) holds for $H$.
\end{proof}

\begin{remark}
	$\NN = \{0, 1, \ldots\} \subseteq\ZZ$, but $\NN\nleq\ZZ$, despite satisfying (a) and (b).
\end{remark}

\chapter{(Left) Coset equivalence} % (fold)

Suppose $G$ is a group and $H\le G$ is a subgroup of $G$. 

\begin{definition}
	We say $x\sim y \pmod H$ if $\inv x y\in H$.
\end{definition}

\begin{proposition}
	$\sim \pmod H$ is an equivalence relation.
\end{proposition}

\begin{proof}\leavevmode
	\begin{itemize}
		\item Reflexivity ($x\sim x$):

		$\inv x x = e\in H$, so $x\sim x$.

		\item Symmetry ($x\sim y \implies y \sim x$):

		We know $\inv x y \in H$. $H$ is closed under inversion, so $H \ni (\inv x y)^{-1} = (\inv y (\inv x)^{-1}) = (\inv y x)$. Thus, $y \sim x$.

		\item Transitivity ($(x\sim y) \land(y \sim z) \implies (x\sim z)$):

		We know $\inv x y \in H$ and $\inv y z \in H$. 
		
		Thus, $H \ni (\inv x y)(\inv y z) = \inv x e z = \inv x z$, so $x\sim z$.
	\end{itemize}
\end{proof}

% subsubsection  (end)
\end{document}
