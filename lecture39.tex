\documentclass[notes.tex]{subfiles}

\begin{document}
\lecture{39}{2016--04--29}

Now we will prove theorem \ref{thm:non-constructible} by induction on $n$.
\begin{proof}
	We proceed by induction on $n$, the length of the construction.

	Base Case: $F = \QQ$ works, ${0, 1}\subseteq F$, $[F:\QQ] = 1 = 2^0$ \checkmark

	Inductive Step: Assume true for constructions of length $\le n$.  We want to handle $\underbrace{P_0\subseteq P_1\subseteq \ldots \subseteq P_n}_{\text{Construction of length }n}\subseteq P_{n+1}$.
	By induction hypothesis, there exists a field $F_n$ with $\QQ\subseteq F_n\subseteq \RR$ such that $\{x, y \text{ coords of } P_n\}\subseteq F_n$ and $[F_n:\QQ] = 2^\ell$.

	\begin{claim}[1]
		Any line through two points of $P_n$ is given by an equation of the form $ax+by+c = 0$ for some $a, b, c\in F_n$.
	\end{claim}
	\begin{proof}[Pf (C1)]
		Say that $(x_0, y_0), (x_1, y_1)$ are the points in $P_n$.
		Then $(x, y)$ is on the line if $(x_1-x_0)(y-y_0) = (x-x_0)(y_1-y_0)$

		$-(y_1 - y_0)x + (x_1 - x_0)y + \text{blah} = 0$, where each of those terms is in $F_n$.
	\end{proof}
	\begin{claim}[2]
		Any circle determined by three points in $P_n$ has equation $(x-h)^2 + (y-k)^2 + r = 0$ for some $h, k , r\in F_n$.
	\end{claim}
	\begin{proof}[Pf (C2)]
		Center point $(x_0, y_0)$, radius points $(x_1, y_1), (x_2, y_2)\in P_n$

		The Radius = $\sqrt{(x_2-x_1)^2 + (y_2-y_1)^2}$
		The equation for the circle becomes 
		\[
			(x-x_0)^2 + (y-y_0)^2 - ((x_2 - x_1)^2  + (y_2-y_1)^2)
		\]
		Where all terms are in $F_n$
	\end{proof}
	There are three cases for the move $P_n\to P_{n+1}$.

	Case 1 (Two lines intersecting at one point):
	\begin{tabin}
		Solve $\begin{cases}
			a_1x + b_1y + c_1 &= 0\\
			a_2x + b_2y + c_2 &= 0
		\end{cases}$ for $(x', y')$ and we leave it to the reader to verify we obtain $x', y' \in F_n$. Thus, $F_{n+1} = F_n$ so we still have $\{x, y \text{ coords of} P_{n+1}\}$ and $[F_{n+1}:Q] = 2^\ell$.
	\end{tabin}
	Case 2 (Line and a circle):
	\begin{tabin}
		Say the line is 
		$ax+by + c = 0\quad\text{(A2)}$ 

		and the circle is
		$(x-h)^2 + (y-k)^2 + r = 0 \qquad\text{(B2)}$.

		Call the simultaneous solutions $(x', y')$ and $(x'', y'')$.

		Solve (A2) for one variable. WLOG, $y = dx + e$.

		Substitute into (B2) to obtain 
		\[
			(x-h)^2 + (dx+e-k)^2 + r =0
		\]
		$x'$ is a root of a degree 2 polynomial $f(x)\in F_n[x]$.

		So $[F_n(x'):F_n]\le 2$, so it's either 1 or 2.
		Hence, $[F_n(x'):\QQ]=[F_n(x'):F_n]\times[F_n:\QQ] = 2^\ell$ or $2^{\ell+1}$ 

		\begin{claim}[3]
			$\{x', y', x'', y''\}\subseteq F_n(x')$ (so $F_{n+1} = F_n(x')$)
		\end{claim}
		\begin{proof}[Pf (C3)]
			$x''\in F_n(x')$\checkmark.
			Why? $x''$ is also a root of $f(x)$. Over $F_n(x'),$ we know that $(x-x')\divides f(x).$ Say $(x-x')\cdot g(x) = f(x)$. Thus, $\deg(g(x)) =1, g(x)\in (F_n(x'))[x]$ So $x''$ is a root of $g(x)\implies x''\in F_n(x')$. 

			But $y' = dx' +e, y'' = dx'' + e \in F_n(x')$\qedhere(C3)
		\end{proof}
		This also completes Case 2
	\end{tabin}
	Case 3 (Two Circles):
	\begin{tabin}
		Fix the two equations A3 and B3. If we subtract B3 from A3, then we get equation C3 $ax+by+c=0$.
		This is handled in a previous case
	\end{tabin}
	This completes the proof.\qedhere(Theorem)
\end{proof}

\begin{corollary}
	Angle trisection cannot be performed by a straight-edge and compass construction.
\end{corollary}
\begin{proof}
	We consider the angle $60^\degree$.
	It suffices to show that $\cos(20^\degree)$ is not constructible

	Why does this rule out trisection?

	Well we can construct $60^\degree$, so if we could trisect it, we could construct  the point $(\cos(20^\degree), \sin(20^\degree))$.

	Why isn't $\cos(20^\degree)$ constructible? 
	We want its minimal polynomial. $\frac{1}{2} = \cos(60^\degree) = \cos 40 \cos 20 - \sin 40 \sin 20$
	Denote $s = \sin 20, c = \cos 20$
	$(2c^2-1)c -2scs = 2c^3 - c - 2c(1-c^2)=4c^3-3c$.
	Hence $c$ is a root of $8x^3 - 6x - 1$.

	Exercise: This polynomial is irreducible in $\QQ[x]$.

	Hence $[\QQ(\cos 20^\degree):\QQ]$ is not a power of 2, so $\cos20$ is not constructible.
\end{proof}
\end{document}