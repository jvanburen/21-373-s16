\documentclass[notes.tex]{subfiles}

\begin{document}

\lecture{7}{2016--01--29}

% \section*{Left Coset Equivalence (Continued)}

$G$ is a group. $H \le G$ a fixed subgroup of $G$.

Given $x, y\in G$, $x\sim y \pmod H$ iff \[
	\inv xy\in H.
\]
Last time: we showed it was an equivalence relation.

What are the equivalence classes of $\sim\pmod H$?
We examine

\begin{align*}
	[x] &= \{y\in G: x\sim y\pmod H\}\\
	&= \{y\in G : \inv xy\in H\}\\
	&= \{y\in G : \exists h \in H (\inv x y = h)\}\\
	&= \{y\in G : \exists h\in H (x(\inv xy)=xh)\}\\
	&= \{y\in G : \exists h \in H (y = xh)\}
\end{align*}
So, $[x]$ is exactly the set \[
	\{xh:h\in H\}.
\]
\begin{notation}
	We write $xH$ to abbreviate the set $\{xh:h\in H\}$.
\end{notation}

\begin{definition}
	The equivalence class $xH$ is called the (left) \kw{coset} of $x$ with respect to $H$.
\end{definition}

\begin{notation}
	The cyclic subgroup of $x$ is denoted by $\langle x\rangle$.
\end{notation}

Examples:
\begin{itemize}
	\item $G = (\ZZ, +)$, $H = n\ZZ = $ multiples of $n$.
	So $H\le G$. For $x\in\ZZ,$ its coset is $\bar x = \{x+h:h\in n\ZZ\} = \{x + nk:k\in\ZZ\}$
	\item $G = S_3 = \{e, \cyc{1 2}, \cyc{1 3}, \cyc{2 3}, \cyc{1 2 3}, \cyc{1 3 2}\}$
	Take $H = \{e, \cyc{1 2 3}, \cyc{1 3 2}\} \le S_3$ (the cyclic subgroup of (1 2 3)).

	So what are the cosets? $eH = \{eh: h\in H\} = \{h : h \in H\} = H$.
	(In general, $eH$ is always just $H$). (Even more generally, $xH = H$ whenever $x\in H$.)

	Another coset is (1 2)$H$.
	Just compute (1 2)$h$ for each $h\in H$.
	Thus \[
		\cyc{1 2}H = \left\{
		\begin{array}{rll}
		 	\cyc{1 2}&e &= \cyc{1 2}\\
			\cyc{1 2}&\cyc{1 2 3} &= \cyc{2 3}\\
			\cyc{1 2}&\cyc{1 3 2} &= \cyc{1 3}\\
		 \end{array}
		 \right\} = \{\cyc{1 2}, \cyc{2 3}, \cyc{1 3}\}
	\]
	We note that $\cyc{1 2}H = \cyc{2 3}H = \cyc{1 3}H$, as each of those
	are in $\cyc{1 2}H$.

	% TODO: MENTION ``EASY'' WAY TO MULTIPLY PERMUTATIONS IN CYCLE NOTATION?

	\item
	$G = S_3$, $K = \langle\cyc{1 3}\rangle  = \{e, \cyc{1 3}\} \le G$.
	Analyze cosets mod $K$.

	Easy coset: $eK = K$.

	For the next coset, choose $\cyc{1 2 3}K$
	\[
		\cyc{1 2 3}K = \left\{
		\begin{array}{ll}
		 	\cyc{1 2 3} \;e &= \cyc{1 2 3}\\
			\cyc{1 2 3} \;\cyc{1 3} &= \cyc{2 3}\\
		 \end{array}
		 \right\} = \{\cyc{1 2 3}, \cyc{2 3}\}
	\]
	Next coset after that is  $\cyc{1 2}K = \{\cyc{1 2}, \cyc{1 3 2}\}$.

	We note that the equivalence classes mod $K$ partition $S_3$, Although they are not all subgroups.
\end{itemize}

In the last two examples, it wasn't a coincidence that each coset was of the same cardinality.

\begin{proposition}
	Suppose $G$ is a group, $H\le G$, and $x\in G$. Then $|xH| = |H|$.
\end{proposition}
\begin{proof}
	We establish a bijection between $H$ and $xH$.

	Define $\varphi:H\to xH$, $\varphi(h) = xh$.

	\begin{claim}[1]
	 	$\varphi$ is surjective.
	 \end{claim}
	 \begin{proof}
	 	Suppose $y \in xH$.

	 	By definition of $xH$, $\exists h\in H$ such that $y=xh$. so $y=\varphi(h)$.\qedhere(C1)
	 \end{proof}
	 \begin{claim}[2]
	 	$\varphi$ is injective.
	 \end{claim}
	 \begin{proof}
	 	Suppose $h_1, h_2\in H$ such that $\varphi(h_1) = \varphi(h_2)$.

		By definition of $\varphi,$ we have $xh_1 = xh_2$. Since $G$ is a group, $x$ has an inverse $\inv x$.

		Thus, $\inv x(xh_1) = \inv x(xh_2)\implies h_1=h_2$ as desired. \qedhere(C2)
	 \end{proof}

	Thus $\varphi$ is a bijection, meaning $|xH| = |H|$ as desired.\qedhere(Prop.)
\end{proof}

\begin{theorem}[Lagrange]\index{Lagrange's Theorem}
	Suppose that $G$ is a finite group and $H\le G$.

	Then $|H|$ divides $|G|$.
\end{theorem}
\begin{proof}
	Left coset equivalence partitions $G$ into $k$ equivalence classes of size $|H|$. \\
\hspace{3em}
	Thus $|G| = k|H|$, as desired.
\end{proof}

\begin{corollary}
	Suppose that $G$ is a finite group and $x\in G$. Then $|x|$ divides $|G|$.
\end{corollary}

\begin{proof}
	Consider $\langle x\rangle$ (the cyclic subgroup generated by $x$).
	$\langle x\rangle = \{e, x, x^2, \ldots, x^{n-1}\}$, where $|x| = n$. $|\langle x\rangle| = n$. Hence $n = |x|$ divides $|G|$.
\end{proof}
\end{document}
